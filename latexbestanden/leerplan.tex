% Documentatie Herfstkamp project: Bouw van een audio-versterker
% Jules Hammenecker
% Vrije Universiteit Brussel - Dept. ELEC
% Augustus 2015

\title{Project Ingenieurswetenschappen: \\ Elektronisch ontwerp van de e-VUBOX \\ Eindtermen}
\author{Vrije Universiteit Brussel}
\date{Versie 08.2015}

\documentclass{article}

	\usepackage[a4paper]{geometry}
	\usepackage[T1]{fontenc}
	\usepackage[dutch]{babel}
	\usepackage{amsmath}
	\usepackage{pdflscape}
		\newtheorem{DIY}{Doe-het-zelf}
	\usepackage{multicol}

	%%% FIGURES %%%

	\usepackage[pdftex]{graphicx}
	\usepackage{caption,subcaption}
	\usepackage{hyperref}
	\graphicspath{ {./figs/} }

\begin{document}
	\maketitle
	In dit document vindt U de eindtermen van het secundair onderwijs van de Vlaamse Overheid die worden aangehaald in het eVUBOX project.

	\section{Vakgebonden eindtermen fysica}

		\paragraph*{F1} grootheden uit onderstaande tabel
			\begin{itemize}
				\item benoemen
				\item de eenheid ervan aangeven
				\item defini\"eren in woorden en met behulp van de formule de eenheid aangeven
				\item het verband leggen tussen deze eenheid en de basiseenheden uit het SI-eenhedenstelsel
				\item de formule toepassen
			\end{itemize}

			\begin{table}[h!]
			\centering
				\begin{tabular}{|l|}
				\hline
					Periode  \\ 
					Frequentie  \\
					Uitwijking van H.T.  \\
					Elektrische spanning  \\
					Elektrische stroomsterkte  \\
					Ohmse weerstand  \\
					Vermogen bij ohmse weerstand  \\
				\hline
				\end{tabular}
				%\caption{caption}
				%\label{tbl:label}
			\end{table}

		\paragraph*{F7} het belang van fysische kennis in verschillende opleidingen en beroepen illustreren
		\paragraph*{F18}	voor een geleider in een gelijkstroomkring het verband tussen spanning, stroomsterkte en weerstand toepassen.
		\paragraph*{F19} de energieomzettingen in elektrische schakelingen met voorbeelden illustreren en het vermogen berekenen
	\section{Vakgebonden eindtermen Wiskunde}
	\paragraph*{1} wiskundetaal begrijpen en gebruiken.
	\paragraph*{2} wiskundige informatie analyseren, schematiseren en structureren.
	\paragraph*{3} eenvoudig mathematiseerbare problemen ontleden (onderscheid maken tussen gegevens en gevraagde, de relevantie van de gegevens nagaan en verbanden leggen ertussen) en vertalen naar een passende wiskundige context.
	\paragraph*{4}	wiskundige problemen planmatig aanpakken (door eventueel hi\"erarchisch op te splitsen in deelproblemen).
	\paragraph*{6}		voorbeelden geven van re\"ele problemen die met behulp van wiskunde kunnen worden opgelost.
	\paragraph*{19}		het begrip afgeleide herkennen in situaties buiten de wiskunde.
	\paragraph*{31}		bij het oplossen van een probleem, waarbij gebruik gemaakt wordt van bestudeerde functionele verbanden, een functievoorschrift, een vergelijking of een ongelijkheid opstellen.
	\paragraph*{32}		tabellen en grafieken bij bestudeerde functies als hulpmiddel gebruiken om functievoorschriften, vergelijkingen en ongelijkheden te interpreteren.
	\section{Specifieke Wiskunde}
	\paragraph*{19}	de bepaalde en de onbepaalde integraal van functies berekenen en ze in concrete situaties gebruiken;

\end{document}